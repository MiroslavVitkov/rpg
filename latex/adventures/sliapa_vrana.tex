% RPG fantasy adventure.
% View with pdflatex sliapa_vrana.tex && evince sliapa_vrana.pdf.
% sir.vorac@gmail.com

\documentclass{article}
\usepackage[T2A]{fontenc}
\usepackage[utf8]{inputenc}
\usepackage[bulgarian]{babel}
%\usepackage{iwona}
%\newenvironment{directspeech}{\itshape}{}
\newcommand{\directspeech}[1]{#1}  % TODO: make this italics.
\begin{document}
{\huge Сляпа врана}
\section{Сър Ленли}
\directspeech{Докато пътувате за някъде си тъй се случва, че заедно се озовавате по мръкване пред страноприемница с тъпото име "Сляпа врана". Точно пък сега хърбав конник зачуква пергамент до вратата.}
Самият служител е неразговорлив, така даваме възможност да се научи кои герои могат да четат.
На пергамента пише на езика на местното кралство че
\directspeech{Сър Ленли - нечуван храбрец и земевладелец по пограничните земи - събира доброволческа армия за свещена война за отмъение.
Има разрешение от краля - пише.
Добре дошли са всички способни да стрелят с лък или да размахват меч, копие или топор.}
Това героите, които го научат, да си го отбележат на персонажния лист някъде - в Бележки, по полетата ...

\section{Вечерята}
Това за които решат да нощуват в страноприемницата.
Тази сцена трябва да е бавна и незастрашаваща, за да е контрастът със следващата най-ярък.
За вечеря има - и нека попитат за това, за да се разсеят -
\directspeech{вино, хляб и свинско печено за пет монети или супа от крачка и царевична каша за една}.
Още няколко пътника вечерят или пият.
Затлъстял мъж в лилава туника вечеря мълчаливо с охранителя си.
Старец налива чаша след чаша от стомна тъмна напитка.
На две маси има групички от по двама и трима войника - познават се по премеринеите движения и препасаните оръжия.
Бедно облечен мъж със съпругата си и дъщерята.
Четири момчета, гордо препасали мечове.
Стаите струват по две моменти леглото.
Дебеля е търговец и може да купи до хиляда монети; продава бижута, както и вълшебни камъни с дребни магии.
Не позволявай тази сцена да се развива повече от петнадесет минути, за да не омръзне с монотонността си.

\section{Среднощ}
Стаите са с по шест сламени дюшека на мазен дървен под.
Всички герои са в една стая, останалото до шест допълнено от персонажи от предишната секция.
Прозорците са по един на стая, с дървен кепенец - жега е и кепенецът е отворен.
Внезапоно в средата на нощта звук на счупено събужда спящите.
Навън са се подредили петима наемници с лети нагръдници.
Току-що са заредили тежките си арбалети, залостили са входната и задната врата, и са метнали по делва с олио и горящ фитил през всяка от капандурите на втория етаж.
Сега дебнат някой да се покаже.
Платено им е да убият Сър Кенедик, който никакъв Сър не е, а крал!
Кенедик е дин от петимата войници - останалтие са личната му охрана.
Притежава кама със скъпоценен камък в помела.
Скопеценният камък може да държи една магия на въздуха - дори от много висока сила - в себе си.
В началото на схватката е зареден с могъща мълния, която вероятно ще убие уцеления и ще зашемети за минути околните.
Войниците носят фина халчеста броня на два слоя и меч. В стаята си държат арбалети, щитове и пътна храна.
Дисагите на краля със скъпоценния товар са захвърлени небрежно в конюшнята, до верния му жребец.

\section{Развръзка}
Би трябвало охраната на краля да се справи дори без помощ от героите.
Ако обаче съществена такава бъде предоставена, Кенедик подарява кинжала си на най-изявилия се (вече мълнията е използвана).
Във всеки случай ако героите помогнат дори и малко, той се представя като Сър Кенедик - дребен благородник - и кани героите да го потърсят ако минават през земите му.
Това също е добре да бъде записано на персонажните листове.
\end{document}
