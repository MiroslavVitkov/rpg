% This file concers character charuin and eviutuion. It;s aim is to be consice anad genre-independent.

\section{Игрови персонаж}

\subsection{Точки за герой}
Броят точки на играч се определя от разказвача според мащаба на приключението.
С точки за герой се закупуват както основните показатели - сила, ловкост, ум и харизма - така и положителни и отрицателни качества.
Някои раси също струват точки за герой за да бъдат избрани.
Уменията, които владее героят, се определят от нивото на показателят му Ум.
\\
\\
\rowcolors{1}{white}{lightgray}
\begin{tabular}{p{3cm} | p{2cm} | p{6cm} | p{3cm}}
Ролева игра          & Точки за герой      & Тактическа игра                   & Точки       \\
гражданин на света   & 20                  & отряд(8-13)                       & 250         \\
герой / гений        & 30                  & бригада(3000-5000)                & 10 000      \\
митично същество     & 50                  & армия(80 000 - 200 000)           & 400 000     \\
стихия               & 100                 & група армии (400 000 - 1 000 000) & 20 000 000  \\
\end{tabular}

\subsection{Качества}
Това са както естествени, така и свръхестествени дарби или проклятия.
Полезните дарби струват по една точка за герой, а вредящите по -1, с разумното ограничение за не повече от три.
Твърде "силни" дарби, като наприемр телепортиране, се препоръчв да се разделят на няколко базови части, всяка от която струва по 1 точка за герой.
Едно възможно разделение от примера е телепортиране - усещане на дестинацията, телепортиране - активиране с мисъл, телепортиране - пренасяне.

\subsection{Показатели}
Рамките на показателите се определят от расата, нормално 1-10 за човеците.
С други думи, не е позволено да се сложи по малко от 1 точка на кой да е показател, нито повече от 10.

\begin{itemize*}
\item {\bfСила} – телосложение, експлозивна сила, статична сила, умение да се концентрира силата на тялото, имунна система
\item {\bfЛовкост} – както пъргавина и гъвкавост на цялото тяло, скорост, така и прецизност във фините задачи
\item {\bfУм} – еродиция, памет, разсъдък, концентрация, наблюдателност, схватливост
\item {\bfХаризма} – увереност, емпатия, изразителност, излъчване
\end{itemize*}

\subsection{Умения}
С \textit{(Ум * 5)} точки разполагаш, за да разпределиш по умения.
Всяко умение е базирано върху някой показател.
Половината от показателя (закръглено надолу) се добавя към ранговете на умението, когато то се ползва.
Повишаване на умение на повече рангове,  отколкото е показателят, струва по три точки за всеки ранг на умението.

\subsection{Носене}
\rowcolors{1}{white}{lightgray}
\begin{tabular}{l | l | l }
Максимален товар, кг & Придвижване, м            & Модификация  \\
Сила                 & Сила + Ловкост + Ловкост  &  0           \\
Сила * 2             & Сила + Ловкост            &  0           \\
Сила * 5             & Сила                      & -5
\end{tabular}


\subsection{Предмети}
Героите започват с един среден, три евтини и колкото си пожелаят много евтини предмети.
Прогресията е много скъп(замък) - скъп(имение) - среден(двуръчен меч) - евтин(меч) - много евтин(щит, брадва).

\subsection{История и биография}
Добре описана история на персонажа помага както на играча да се вживее, така и на разказвача да следи целите на групата.
В случай, че не ти стигат идеите, пробвай да отговориш на следните въпроси.
\begin{itemize*}
\item{Какво искаш от утрешния ден, от идните години, от живота?}
\item{Какво най-много цениш в хората? А кое мразиш най-много?}
\item{Опиши семейството си.}
\item{Опиши мястото, където си израстнал.}
\item{Опиши двама приятели или познати от миналото си.}
\item{Защо в момента се занимаваш с това?}
\item{Коя религия следваш? Защо?}
\item{Имаш ли прякор? Как го получи?}
\item{Отнемал ли си човешки живот? А животински? Хареса ли ти?}
\end{itemize*}
