% This file concers character charuin and eviutuion. It;s aim is to be consice anad genre-independent.

\chapter{Игрови персонаж}

\section{Точки за герой}
С 'точки за герой' се купуват раса, показатели и качества.
Началните точки се определят според сценария - 20 за обикновен човек, 30 за 'герой', 50 за митично същество, 100 за езически бог.


\section{Раса}
Тук думата се ползва в значение 'биологичен вид'.
Нека разгледаме човеците за пример.
\begin{multicols}{2}
\race{Човек}{0}{}{1 - 10}{1 - 10}{1 - 10}{1 - 10}{няма}{няма}
(0) означава, че расата струва 0 точки за герой.
Показателите означават, че играчът е длъжен да сложи поне 1 точка на всеки от тях(0 точки означава инвалид), но не повече от 10.
Качества са свръхестествени особености на расата, които всеки неин представител притежава.
Расови умения: могат да бъдат избрани две от списъка - те започват на 5 точки без да е нужно да се покривт изисквния за това и без да се заплащат от (Ум*5).
Човеците са една от малкото раси, които нямат расови(т.е. културно наложени) умения.
\end{multicols}


\section{Показатели}
Наричани още primary attributes, защото останалите характеристики се определят от тях.
Ето някои примери:
\begin{itemize*}
\item {\bfСила} – щети в ръкопшен бой, мъкнене на багаж, носене на тежка пушка, тичане, оцеляване след рана
\item {\bfЛовкост} – нападение, париране, отключване, промъкване, забелязване
\item {\bfУм} – магия, компютри, брой умения
\item {\bfХаризма} – преговаряне, усещне на лъжи, изтезаване
\end{itemize*}


\section{Умения}
Точките за умения са \textit{(Ум * 5)} и се увеличават или намаляват ако Ум се промени.
Всяко умение е базирано върху някой показател и не може да бъде повишавано на повече точки от него.
\\
\\
Глава Таблици дава някои примери за умения.
Това са само примери и не бива да ограничават въображението!


\section{Качества}
Шапка за всичко останало.
Искаш да започнеш с много пари? Качество 'богат'.
Искаш да си извадиш бира от хлдилника без да ставаш? Качество 'телекинеза'.
Искаш 1 допълнителна точка за герой? Качество късоглед.
\\
\\
Полезните качества струват по 1 точка за герой, а вредящите(до три) по -1.
Твърде "силни" дарби се разделят на няколко базови части, всяка от която струва по 1 точка за герой.
Примерно телепортиране: визулизирне на дестинацията, пренасяне, активиране с мисъл.
Или сакат: липсваща китка, липса на ръка.
\\
\\
Отново глава Таблици дава някои идеи.


\section{Носене}
\rowcolors{1}{white}{lightgray}
\begin{tabular}{l | l | l | l }
Максимален товар, кг & Придвижване, м/ход       & Модификация  \\
Сила                 & Сила + Ловкост + Ловкост & 0            \\
Сила * 2             & Сила + Ловкост           & 0            \\
Сила * 5             & Сила                     & -5
\end{tabular}
\\
Очевидно хвърлянето на раницата преди битка е добра идея, както и язденето на дълги разстояния.


\section{Предмети}
Героите започват с един среден, три евтини и колкото си пожелаят много евтини предмети.
Прогресията е много скъп(замък) - скъп(имение) - среден(двуръчен меч) - евтин(меч) - много евтин(щит, брадва).

\section{Биография}
Ето някои идеи за историята на персонажа:
\begin{itemize*}
\item{Какво искаш от утрешния ден, от идните години, от живота?}
\item{Какво най-много цениш в хората? А кое мразиш най-много?}
\item{Опиши семейството си.}
\item{Опиши мястото, където си израстнал.}
\item{Опиши двама приятели или познати от миналото си.}
\item{Защо в момента се занимаваш с това?}
\item{Коя религия следваш? Защо?}
\item{Имаш ли прякор? Как го получи?}
\item{Отнемал ли си човешки живот? А животински? Хареса ли ти?}
\end{itemize*}
