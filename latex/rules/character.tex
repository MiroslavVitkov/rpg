% This file concers character charuin and eviutuion. It;s aim is to be consice anad genre-independent.

\section{Игрови персонаж}
\emph{Ел'тлменел от селото Нлимна има резки чети за елф, и определено неестествен ръст на неговите метър и деведесет.
Олекотената ризница от еленски рога не приквива силните му ръце и крака, а копието и лъка допълват картината на ловец/войн, само тъмнозеленото ленено наметало остава.
Менел, както го наричат не-елфите навсякъде, е израстнал в гори, не умее да чете и е страшно глупав, когато става дума за светски дела.}

Героите на нашето приключение трябва да бъдат описани по прост, но съдържателен начин.
Това става с \textbf{дарби, показатели и умения}.

\subsection{Дарби}
Това са свойства на перонажа, с които той започва, и не се очаква да могат да бъдат променени някога.
Примери вкючват “Интуитивна ориентация в тъмното”, “Огнена кръв(раните не кървят, а веднага се затварят)”, “Слепота”.
Персонажът Нафарфорий/Пешо се създава с {\bfедно положително} == помагащо му качество.
Или до {\bfоще три}, като за всяко допълнително “хубаво” качество, си харесваме по едно “лошо”. Стига се цупи де, лошите качества са крайно забавни за разиграване!

\subsection{Показатели}
Разполагаш с \textbf{30 точки}, които да разпределиш по показателите. Те се променят рядко.
\begin{itemize}
\item {\bfСила} – телосложение, експлозивна сила, статична сила, умение да се концентрира силата на тялото, имунна система
\item {\bfЛовкост} – както пъргавина и гъвкавост на цялото тяло, скорост, така и прецизност във фините задачи
\item {\bfУм} – еродиция, памет, разсъдък, концентрация, наблюдателност, схватливост
\item {\bfХаризма} – увереност, емпатия, изразителност, излъчване
\end{itemize}

\subsection{Умения}
Начални точки и развиване  \\
С \textit{(Ум * 5)} точки разполагаш, за да разпределиш по умения.
Уменията, също така, се повишават с точки опит по време на игра.
\\
\\
Ниво на владеене  \\
Нивото на владеене на всяко умение се оценява по скала от 0 до 15, като 15 е специална стойност.
Всяко умение е базирано върху някой показател.
Половината от показателя (закръглено надолу) се добавя към ранговете на умението, когато то се ползва.
Повишаване на умение на повече рангове,  отколкото е показателят, струва по две точки за всеки ранг на умението.

\subsection{Носене}
\rowcolors{1}{white}{lightgray}
\begin{tabular}{l | l | l }
Максимален товар, кг & Придвижване, м            & Модификация  \\
Сила                 & Сила + Ловкост + Ловкост  &  0           \\
Сила * 2             & Сила + Ловкост            &  0           \\
Сила * 5             & Сила                      & -5
\end{tabular}


\subsection{Предмети}
Героите започват с един среден, три евтини и колкото си пожелаят много евтини предмети.
Прогресията е много скъп(замък) - скъп(имение) - среден(двуръчен меч) - евтин(меч) - много евтин(щит, брадва).

\subsection{История и биография}
Добре описана история на персонажа помага както на играча да се вживее, така и на разказвача да следи целите на групата.
В случай, че не ти стигат идеите, пробвай да отговориш на следните въпроси.                       \\
1) Какво искаш от утрешния ден, от идните години, от живота?                                      \\
2) Какво най-много цениш в хората?                                                                \\
3) А кое мразиш най-много?                                                                        \\
4) Опиши семейството си.                                                                          \\
5) Опиши мястото, където си израстнал.                                                            \\
6) Опиши поне двама приятели или познати от миналото си, които не са други играови персонажи.     \\
7) Защо в момента се занимаваш с това?                                                            \\

\subsection{Развитие}
През определени периоди време се начислява промяна в способностите на персонаж.
Губи се една точка на някое умение, което не е ползвано последната сесия.
Получаса се една точка на някое умение (може да е същото каот изгубеното), което героят е ползвал от последното начисляване.
Получава още една точка най-много допринеслият през последната сесия.
Получава още една точка най-много допринеслият от последното начисляване.

