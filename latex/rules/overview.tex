\chapter{Зарове}

Наричаме 'd*' зар с числата {-5,-3,-1,2,4,6}.

За да се провери дали дадено действие успява, се изчислява (умение + d* - трудност).
Често 'трудност' е някое умение или показател на опонента.

\rowcolors{1}{white}{lightgray}
\begin{tabular}{ p{2cm} | p{2cm} |  p{2cm} | p{2cm} | p{6cm} }
Абстракция        & ръкопашен бой                & Стрелба                 & Магия                & Общуване      \\
повлияване някому & боравене(Л)                  & боравене(Л)             & умение(У)            & пазарене(цени, условия)(Х), молене(призоваване на съпричастност)(Х), принуждаване(сплашване, заповядване)  \\
предпазване       & боравене(Л)                  & Ловкост, броня(С)       & показател            & емпария(разгадаване на подбуди)(Х), дисциплина(виздуржане на импулсивни решения), Сила(телесни трансформации), Ловкост(отбягване на проектили), Ум(видения), Харизма(внушения)  \\
ниво на успех     & разлика(Л)                   & разлика(Л)              & разлика(У)           & отношения(Х)  \\
ниво на неуспех   & разлика(Л)                   & разлика(Л)              & разлика(У)           & отношения(Х)  \\
обсег             & по-дългото оръжие удря първо & от (Сила/2) до (Сила*2) & (Ум*2)               & Ум * 2 метра  \\
собствен ресурс   & здраве(С)                    & здраве(С)               & здраве(С), умение(У) & отношения(Х)  \\
вражески ресурс   & здраве(С)                    & здраве(С)               & -                    & отношения(Х)  \\
\end{tabular}
