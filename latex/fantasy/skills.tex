\chapter{Умения}
\section{Боравене}
Ръкопашните умения за ползват за удряне и париране, за подбор на качествено оръжие и за полевата му поддръжка.
Уменията с обсадни оръжия се ползват за сглобяване, разглобяване и командване на стрелбата. 
\subsection{Ръкопашни}
\Tree[ .Невъоръжен [ .Борба ] [ .Удари ] [ .{Импровизирани оръжия} ] ]
\Tree[ .Фехтовка [ .{Нож, Кама} [ .Гладиус ] ] [ .{Едноръчен меч} [ .Рапира ] [ .Сабя ] ] [ .{Дълъг меч} [ .Цвайхендер ] [ .Есток ]  ]  ]
\vskip 0.5cm
\Tree[ .Копие [ .Тояга ] [ .{Късо копие} ] [ .{Дълго копие} ] [ .Алебарда ] [ .Пика ] ]
\Tree[ .Топор [ .Брадвичка ] [ .{Датска брадва} ] [ .Бардич ] ]
\hskip 1cm
\Tree[ .Чук [ .Тояжка ] [ .{Боен чук} ] [ .Боздуган ] [ .{Двуръчен боздуган} ] ]

\subsection{Метателни}
\Tree[ .{Метателен нож} [ .{Метателна брадва} ] ]
\Tree[ .{Метателно копие} ]

\subsection{Стрелящи}
\Tree[ .Балистика [ .Лък ] [ .Арбалет ] [ .Прашка ] [ .Балиста [ .Катапулт [ .Требучет ] ] ] ]

\section{Езици}
\subsection{Човешки езици}
\begin{itemize}[topsep=-0cm, partopsep=0cm, parsep=0cm, itemsep=0cm]
\item{Терейски - човешката инвазия от юг донесла и езика си}
\item{Каланорски - същият човешки език, еволюирал в друга среда}
\item{Спернт - изкуствен език, базиран на човешкия; лесен за учене; говорен в Добра надежда}
\item{Дабарски - сищия този език, в оригиналната форма когато пристигнал на Ендивал}
\item{Злотски - езикът на местното племе около Таури - родени търговци, премесен с Дабарски}
\end{itemize}
\Tree[ .Дабарски [ .Терейски [ .Спернт ] ] [ .Каланорски ] [ .Злотски ] ]

\subsection{Здрачнически езици}
\begin{itemize}[topsep=-0cm, partopsep=0cm, parsep=0cm, itemsep=0cm]
\item{Здрач - говорен в Хайсея}
\item{Староздрачнически - един от езиците по старите земи на здрачниците; говорен само от някои}
\item{Нов здрач - език изцяло със жестове}
\item{Хаски - друг популярен търговски език}
\end{itemize}
\Tree[ .Староздрачнически [ .Здрач [ .Хаски ] ] [ .{Нов здрач} ] ]

\subsection{Върколашки езици}
\begin{itemize}[topsep=-0cm, partopsep=0cm, parsep=0cm, itemsep=0cm]
\item{Фелтеарски}
\item{Лиднарски}
\item{Креолски}
\item{Грефрамски}
\item{Грефраски}
\item{Старовърколашки}
\item{Старобурхски}
\item{Бурхски}
\item{Третски - общ език между нежду наемници и други паравоенни организации из Ендивал}
\end{itemize}
\Tree[ .Старовърколашки [ .Старобурхски [ .Бурхски [ .Третски ] ] ] [ .Грефрамски [ .Фелтеарски ] [ .Лиднарски ] [ .Креолски ] [ .Грефкраски ]  ] ]

\subsection{Елфически езици}
\begin{itemize}[topsep=-0cm, partopsep=0cm, parsep=0cm, itemsep=0cm]
\item{Староелфически}
\item{Елфически}
\item{Даенлински}
\end{itemize}
\Tree[ .Староелфически [ .Елфически ] [ .Даенлински ]  ]

\subsection{Джуджешки езици}
\begin{itemize}[topsep=-0cm, partopsep=0cm, parsep=0cm, itemsep=0cm]
\item{Храфкромдретппелски}
\item{Голмхемрумхерски}
\item{Грумдерхентлестонски}
\item{Краунаумхески}
\item{Църковен джудежешки}
\item{Староджуджешки}
\end{itemize}
\Tree[ .Староджуджешки [ .{Църковен джуджешки} ] [ .Голмхемрумхерски [ .Храфкромдретппелски ] [ .Грумдерхентлестонски ] [ .Краутнаумхески ] ] ]

\subsection{Други езици}
\begin{itemize}[topsep=-0cm, partopsep=0cm, parsep=0cm, itemsep=0cm]
\item{Латински - преоткрит по необходимост заради количеството книги, донесени с човешката инвазия}
\end{itemize}


%\section{Общуване}
%Лъжене (Харизма)                  \\
%Сплашване (Харизма)               \\
%Изтезаване/Разпитване (Харизма)   \\
%Емпатия (Харизма)                 \\
%Четене по устните (Харизма)       \\
%Просене (Харизма)                  \\
%
%\section{Академични}
%Език(Ум) - староендивалски, гоблински, северно върколажко кралство и т.н.  \\
%Наука(Ум) - алхимия, астрономия, геометрия, оптика и т.н.  \\
%Тактика (Ум)                      \\
%
%\section{Схватки}
%Оръжие(Ловкост) - нож, меч, брадва, дълъг меч, лък и т.н.  \\
%Военна машина* (Ум)               \\
%
%\section{Придвижване}
%Плуване (Сила)                    \\
%Кaтерене (Сила)                   \\
%
%\section{Занаяти}
%Билкарсто (Ум)                    \\
%Ковачество(С)                     \\
%Грънчарство(Л)                    \\
%Изработка на лъкове(Л)            \\
%Тъкане, шиене и бродиране(Л)      \\
%
%%\section{Епични провали}
%%Ако при мятане на д6* се падне %5 или +6, се мята отново.
%%Ако случайно се падне същото число, то се взема предивд и се продължава с мятането.
%%Ако крайният резултат е <0, имаме епичен провал, честито!  \\
%
%%При атака с оръжие           \\
%%- да го изтърве              \\
%%- да се сепъне               \\
%%- да се наръга               \\
%%- да се счупи                \\
%%- да удари някого другиго    \\
%%- ми нищо не се случва       \\
