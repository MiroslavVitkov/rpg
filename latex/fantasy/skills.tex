
\section{Умения}
Означените със звездичка умения са събирателни понятия.
Например може да се избере \textit{елфически} език или оръжие \textit{брадва} или наука \textit{алхимия}.
\subsection{Общуване}
Лъжене (Харизма)                  \\
Сплашване (Харизма)               \\
Изтезаване/Разпитване (Харизма)   \\
Емпатия (Харизма)                 \\
Четене по устните (Харизма)       \\
Молене (Харизма)                  \\

\subsection{Академични}
Език* (Ум)                        \\
Наука* (Ум)                       \\
Тактика (Ум)                      \\

\subsection{Схватки}
Оръжие* (Ловкост)                 \\
Военна машина* (Ум)               \\

\subsection{Придвижване}
Плуване (Сила)                    \\
Кaтерене (Сила)                   \\

\subsection{Занаяти}
Билкарсто (Ум)                    \\
Ковачество(С)                     \\
Грънчарство(Л)                    \\
Изработка на лъкове(Л)            \\
Тъкане, шиене и бродиране(Л)      \\

\subsection{Епични провали}
Ако при мятане на д6* се падне -5 или +6, се мята отново.
Ако случайно се падне същото число, то се взема предивд и се продължава с мятането.
Ако крайният резултат е <0, имаме епичен провал, честито!  \\

При атака с оръжие           \\
- да го изтърве              \\
- да се сепъне               \\
- да се наръга               \\
- да се счупи                \\
- да удари някого другиго    \\
- ми нищо не се случва       \\
