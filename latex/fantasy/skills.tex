\chapter{Умения}
\section{Боравене}
Ръкопашните умения за ползват за удряне и париране, за подбор на качествено оръжие и за полевата му поддръжка.
Уменията с обсадни оръжия се ползват за сглобяване, разглобяване и командване на стрелбата. 
\subsection{Ръкопашни}
\Tree[ .Невъоръжен [ .Борба ] [ .Удари ] [ .{Импровизирани оръжия} ] ]
\Tree[ .Фехтовка [ .{Нож, Кама} [ .Гладиус ] ] [ .{Едноръчен меч} [ .Рапира ] [ .Сабя ] ] [ .{Дълъг меч} [ .Цвайхендер ] [ .Есток ]  ]  ]
\vskip 0.5cm
\Tree[ .Копие [ .Тояга ] [ .{Късо копие} ] [ .{Дълго копие} ] [ .Алебарда ] [ .Пика ] ]
\Tree[ .Топор [ .Брадвичка ] [ .{Датска брадва} ] [ .Бардич ] ]
\hskip 1cm
\Tree[ .Чук [ .Тояжка ] [ .{Боен чук} ] [ .Боздуган ] [ .{Двуръчен боздуган} ] ]

\subsection{Метателни}
\Tree[ .{Метателен нож} [ .{Метателна брадва} ] ]
\Tree[ .{Метателно копие} ]

\subsection{Стрелящи}
\Tree[ .Балистика [ .Лък ] [ .Арбалет ] [ .Прашка ] [ .Балиста [ .Катапулт [ .Требучет ] ] ] ]

\section{Езици}
\subsection{Човешки езици}
\begin{itemize}[topsep=-0cm, partopsep=0cm, parsep=0cm, itemsep=0cm]
\item{Терейски - човешката инвазия от юг донесла и езика си}
\item{Каланорски - същият човешки език, еволюирал в друга среда}
\item{Спернт - изкуствен език, базиран на човешкия; лесен за учене; говорен в Добра надежда}
\item{Дабарски - сищия този език, в оригиналната форма когато пристигнал на Ендивал}
\item{Злотски - езикът на местното племе около Таури - родени търговци, премесен с Дабарски}
\end{itemize}
\Tree[ .Дабарски [ .Терейски [ .Спернт ] ] [ .Каланорски ] [ .Злотски ] ]

\subsection{Здрачнически езици}
\begin{itemize}[topsep=-0cm, partopsep=0cm, parsep=0cm, itemsep=0cm]
\item{Здрач - говорен в Хайсея}
\item{Староздрачнически - един от езиците по старите земи на здрачниците; говорен само от някои}
\item{Нов здрач - език изцяло със жестове}
\item{Хаски - друг популярен търговски език}
\end{itemize}
\Tree[ .Староздрачнически [ .Здрач [ .Хаски ] ] [ .{Нов здрач} ] ]

\subsection{Върколашки езици}
\begin{itemize}[topsep=-0cm, partopsep=0cm, parsep=0cm, itemsep=0cm]
\item{Фелтеарски}
\item{Лиднарски}
\item{Креолски}
\item{Грефрамски}
\item{Грефраски}
\item{Старовърколашки}
\item{Старобурхски}
\item{Бурхски}
\item{Третски - общ език между нежду наемници и други паравоенни организации из Ендивал}
\end{itemize}
\Tree[ .Старовърколашки [ .Старобурхски [ .Бурхски [ .Третски ] ] ] [ .Грефрамски [ .Фелтеарски ] [ .Лиднарски ] [ .Креолски ] [ .Грефкраски ]  ] ]

\subsection{Елфически езици}
\begin{itemize}[topsep=-0cm, partopsep=0cm, parsep=0cm, itemsep=0cm]
\item{Староелфически}
\item{Елфически}
\item{Даенлински}
\end{itemize}
\Tree[ .Староелфически [ .Елфически ] [ .Даенлински ]  ]

\subsection{Джуджешки езици}
\begin{itemize}[topsep=-0cm, partopsep=0cm, parsep=0cm, itemsep=0cm]
\item{Храфкромдретппелски}
\item{Голмхемрумхерски}
\item{Грумдерхентлестонски}
\item{Краунаумхески}
\item{Църковен джудежешки}
\item{Староджуджешки}
\end{itemize}
\Tree[ .Староджуджешки [ .{Църковен джуджешки} ] [ .Голмхемрумхерски [ .Храфкромдретппелски ] [ .Грумдерхентлестонски ] [ .Краутнаумхески ] ] ]

\subsection{Други езици}
\begin{itemize}[topsep=-0cm, partopsep=0cm, parsep=0cm, itemsep=0cm]
\item{Латински - преоткрит по необходимост заради количеството книги, донесени с човешката инвазия}
\end{itemize}


\section{Магия}
Хвърля се Умение + d* срещу зададена трудност.
При успех, често разликата определя силата на ефекта.
При неуспех, магьосника търпи микроинсулт за толкова щети, колкото е разликата.

Магически кръгове могат да изричат заклинания синхронизирано.
Трудността на заклинанието се разпределя между магьосниците както те решат.
След това всеки хвърля срещу собственото парче трудност.
Ако някой се провали, изричанеот се проваля и всеки търпи щети като от най-лошото хвърляне..

За всяко настоящо поддържано заклинание от същата школа магия, умението се намаля с 5 точки временно.

Комбинирани заклинания се изричат с най-ниското от уменията във всички школи, но поддържането им отнема точки от всички школи.

\rowcolors{1}{white}{lightgray}
\begin{tabular}{c | c}
Обсег         & Модификация  \\
Ум * 2, м     & -  \\
Ум * 200, м   & -5  \\
Ум * 20, км   & -10  \\
Ум * 2000, км & -15  \\
\end{tabular}
\vspace{1cm}

\textbf{Отвъд}
\begin{itemize*}
  \item{усещане на немъртви - 5}
  \item{изтляване - съществото умира след месец  докато се поддържа магията - 10}
  \item{призоваване на призрак за въпроси - 10}
  \item{връщане на душата в тяло - за възкресяване или вдигане на немъртъв слуга - 10}
  \item{немъртър слуга - успех от хвърлянето по показатели, следва вербални команди - 10}
  \item{превръщане в лич - 30}
\end{itemize*}


\textbf{Плът}
\begin{itemize*}
  \item{лекуване на рана - 5 + размер на раната}
  \item{убиване - 15 + Сила на опонента}
  \item{зараза - 10}
\end{itemize*}


\textbf{Гадаене}
\begin{itemize*}
  \item{лекуване на рана - 5 + размер на раната}
  \item{убиване - 15 + Сила на опонента}
  \item{зараза - 10}
\end{itemize*}


\vspace{1cm}
\textbf{Гадание / Пророчество (в пространството и времето)}
\begin{itemize*}
  \item{усещане позицията на предмет след докосване, до няколко километра - 5}
  \item{усещане името на същество - 5}
  \item{виждане на далеч - 5}
  \item{светлина - 5}
  \item{живо оръжие (бoравенето става равно на умението гадаене) - 10}
  \item{уцелване отвъд прикритие (атака = умение) - 10}
  \item{виждане през стени - 10}
  \item{усещане на лъжи - 10}
  \item{виждане в тъмнина - 10}
  \item{посока към най-близкия град/път/река - 10}
  \item{предсказване на удари - +умение към боравене - 15}
\end{itemize*}


\vspace{1cm}
\textbf{Огън}
\begin{itemize*}
  \item{запалване на много свещи - 5}
  \item{запалване на дрехи - 5}
  \item{предпазване от жега - 5}
  \item{огнено копие - Щ = умение, високи, 10}
  \item{палене на колиба - 10}
  \item{запалване на човек - 10}
  \item{защита от огън - 10}
  \item{огнено кълбо - <умение> радиус - 15}
  \item{запалване на тълпа - 15}
  \item{защито от огън на тълпа, 15}
  \item{запалване на град - 20}
  \item{езеро от лава - 20}
  \item{защита от огън на град - 20}
\end{itemize*}

Бъдещи идеи:  \\
Вятър (Ум)        - мъгла, ниво 2 мълния щяти по Ум, високи.                                         \\
Форма (Ум)        - превръщане на предмети или същества в други, промяна на показатели и размери     \\
Отвъд (Ум)        - разговаряне с души, вдигане на мъртви.                                           \\
Създаване (Ум)    - храна и вода, предмети и създания.                                               \\
Миъл (Ум)         - отношение и спомени                                                              \\
Tелепортация (Ум) - пътуване в този и други светове, с товари                                        \\


%\section{Общуване}
%Лъжене (Харизма)                  \\
%Сплашване (Харизма)               \\
%Изтезаване/Разпитване (Харизма)   \\
%Емпатия (Харизма)                 \\
%Четене по устните (Харизма)       \\
%Просене (Харизма)                  \\
%
%\section{Академични}
%Език(Ум) - староендивалски, гоблински, северно върколажко кралство и т.н.  \\
%Наука(Ум) - алхимия, астрономия, геометрия, оптика и т.н.  \\
%Тактика (Ум)                      \\
%
%\section{Схватки}
%Оръжие(Ловкост) - нож, меч, брадва, дълъг меч, лък и т.н.  \\
%Военна машина* (Ум)               \\
%
%\section{Придвижване}
%Плуване (Сила)                    \\
%Кaтерене (Сила)                   \\
%
%\section{Занаяти}
%Билкарсто (Ум)                    \\
%Ковачество(С)                     \\
%Грънчарство(Л)                    \\
%Изработка на лъкове(Л)            \\
%Тъкане, шиене и бродиране(Л)      \\
%
%%\section{Епични провали}
%%Ако при мятане на д6* се падне %5 или +6, се мята отново.
%%Ако случайно се падне същото число, то се взема предивд и се продължава с мятането.
%%Ако крайният резултат е <0, имаме епичен провал, честито!  \\
%
%%При атака с оръжие           \\
%%- да го изтърве              \\
%%- да се сепъне               \\
%%- да се наръга               \\
%%- да се счупи                \\
%%- да удари някого другиго    \\
%%- ми нищо не се случва       \\






