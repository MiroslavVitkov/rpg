\chapter{Езици}
\section{Човешки езици}
\begin{itemize}[topsep=-0cm, partopsep=0cm, parsep=0cm, itemsep=0cm]
\item{Терейски - човешката инвазия от юг донесла и езика си}
\item{Каланорски - същият човешки език, еволюирал в друга среда}
\item{Спернт - изкуствен език, базиран на човешкия; лесен за учене; говорен в Добра надежда}
\item{Дабарски - сищия този език, в оригиналната форма когато пристигнал на Ендивал}
\item{Злотски - езикът на местното племе около Таури - родени търговци, премесен с Дабарски}
\end{itemize}
\Tree[ .Дабарски [ .Терейски [ .Спернт ] ] [ .Каланорски ] [ .Злотски ] ]

\section{Здрачнически езици}
\begin{itemize}[topsep=-0cm, partopsep=0cm, parsep=0cm, itemsep=0cm]
\item{Здрач - говорен в Хайсея}
\item{Староздрачнически - един от езиците по старите земи на здрачниците; говорен само от някои}
\item{Нов здрач - език изцяло със жестове}
\item{Хаски - друг популярен търговски език}
\end{itemize}
\Tree[ .Староздрачнически [ .Здрач [ .Хаски ] ] [ .{Нов здрач} ] ]

\section{Върколашки езици}
\begin{itemize}[topsep=-0cm, partopsep=0cm, parsep=0cm, itemsep=0cm]
\item{Фелтеарски}
\item{Лиднарски}
\item{Креолски}
\item{Грефрамски}
\item{Грефраски}
\item{Старовърколашки}
\item{Старобурхски}
\item{Бурхски}
\item{Третски - общ език между нежду наемници и други паравоенни организации из Ендивал}
\end{itemize}
\Tree[ .Старовърколашки [ .Старобурхски [ .Бурхски [ .Третски ] ] ] [ .Грефрамски [ .Фелтеарски ] [ .Лиднарски ] [ .Креолски ] [ .Грефкраски ]  ] ]

\section{Елфически езици}
\begin{itemize}[topsep=-0cm, partopsep=0cm, parsep=0cm, itemsep=0cm]
\item{Староелфически}
\item{Елфически}
\item{Даенлински}
\end{itemize}
\Tree[ .Староелфически [ .Елфически ] [ .Даенлински ]  ]

\section{Джуджешки езици}
\begin{itemize}[topsep=-0cm, partopsep=0cm, parsep=0cm, itemsep=0cm]
\item{Храфкромдретппелски}
\item{Голмхемрумхерски}
\item{Грумдерхентлестонски}
\item{Краунаумхески}
\item{Религиозен джудежешки}
\item{Староджуджешки}
\end{itemize}
\Tree[ .Староджуджешки [ .Религиозен джуджешки ] [ .Голмхемрумхерски [ .Храфкромдретппелски ] [ .Грумдерхентлестонски ] [ .Краутнаумхески ] ] ]

\section{Други езици}
\begin{itemize}[topsep=-0cm, partopsep=0cm, parsep=0cm, itemsep=0cm]
\item{Латински - преоткрит по необходимост заради количеството книги, донесени с човешката инвазия}
\end{itemize}
