1\section{Бойна система}
\emph{Копитата тракат, капрата проскърцва. Ръми втори ден вече и макар
промазаната качулка да пази лицето ми, ботушите ми са пълни с вода и джвакат.
Сют е кочияш на работа и пияница в другото време, вече от ден не сме си
казвали нищо, нека си дърпа юздите. \\
В гората напред край пътя – движение. \\
“СПРИ!” викам. Фургонът спира, конете цвилят. Две къси копия прелитат,
убиват единия кон. Крадците връхлитат. Хващам щита от куката и скачам,
халчестата ризница дрънчи, вадя меча. Вече замахва нож, устремявам щита
към главата му и сека към коленете. Удрям и го повалям, втори ми е странично
и опитва да забоде ножа в ребрата ми, но ризницата пази. Убивам го. От
капрата връхлита секирата, отсяква главата на Сют.}

\vspace{5mm} \indent
От бойната система се очаква да е както проста и лесна, непречеща на разказването на историята, така и детайлна и категорична, позволяваща вглъбяване в динамиката на онези кратки моменти, в които страшно смелият ни герой е с единия крак в гроба.
В случай на ръкопашен бой се разглеждат отрязъци от по няколко такта на сърцето. За такова време е достатъчно да замахнеш по арка с голямо оръжие, да нанесеш няколко удара с дръжка на пистолет или рапира, да извадиш стрела от колчана.
За престрелки се разглеждат отрязъци от порядъка на сто такта на сърцето. Това отразява търсене на укритие, презареждане на арбалет или побиване на стрели в земята за лък, дебнене дупка в прикриващия огън при огнестрелни оръжия.
Действието се случва на най фината/бързата скала, присъстваща на бойното поле.
Например, когато тичащата пехота е на разстояние 300 метра от срелците с дълъг лък, играем по гранулярността за престрелки.  Бронираната милиция изминава разстоянието, стрелците правят 6 залпа на регулярни разстояние във времето. След това настава клане и играем по гранулярността за меле. Впрочем и двете групи са сега изморени, тъй каот са прекарали цял дълъг хот в интензивно натоварване, и търпят наказание от -5.

\subsection{Тактове време}
\subsection{Ръкопашни дистанции}
\begin{enumerate}
\item{директна (борба)}
\item{близка (нож, невъоръжен, къс меч)}
\item{средна (меч 80см, повечето оръжия за една ръка)}
\item{дълга (дълъг меч, копие, двуръчен меч)}
\item{отдалечена (пика)}
\item{извън ръкопашен обхват (стрелящи неща)}
\end{enumerate}
Инициатива за първия удар има този с по-дълго оръжие (очевидно)
Извън оптималната си дистанция, оръжията са на -5 атака и половин щети.
Придвижване между дистанции се извършва след успешно хвърляне (независимо нападение или защита). Придвижването е или една категория в произволна посока или произволен брой категории, без да се преминава през заплашвани от някой враг дистанции. 

Пример:
\emph{Пешо, с копие и нож, напада Жоро, с меч.
Пешо връхлита и напада с копието. Хвърля боравене + d6* < боравене на Жоро. Жоро е отбягнал атаката и може да промени дистанцията. Пешо заплашва дълга (копие една ръка) и близка(нож). Жоро може да избере близка, средна, дълга, отдалечена или да избяга.
}

\subsection{Стрелба}
Защитата на движеща се цел се равнява на нейната ловкост.

\subsection{Оръжия}
Една характеристика на оръжията е обсегът от Сила, за който са ефективни. Ако персонажът има по-малко от минималната сила, разликата става наказание върху боравенето с това оръжие. От друга страна, Сила, по-висока от максималната, не допринася за по-високи щети.
Друга характеристика са щетите на оръжието. Условно наричаме “ниски” щети от малки оръжия за една ръка, “средни” - тези на по-големите едноръчни оръжия и “високи” тези на повечето двуръчни оръжия.  Самото оръжие може да има бонус или наказание върху щетите си, редом с качествата на оръжието.
Обсегът на оръжията има различен смисъл за ръкопашни и далечни оръжия. При ръкопашните оръжия, това е оптималната дистанция. На всички по-близки дистанции оръжието има наказание от -5 боравене и половин щети. За далечни оръжия, обсегът е максималното разстрояние, на което лесно може да се улучи цел. На по-голямо от това разстояние атаката е с -5 наказание, на по-голямо от два пъти това разстояние става -10 и на от три до четири пъти това разстояние е на -15. Чести стойности за обсег са (Сила/2) – метателен нож, (Сила) – метателно копие, (Сила х 2) – лък .
Също така, оръжията могат да имат някои от следните свойства.
%\being{enumerate}
%\item{повратливост (обикновено мечове; покрива всички по-близки дистанции от максимлната)}
%\item{бронебойност I (чукове или леки клинове, като стрели и копия; намаля абсорбацията на половина)}
%\item{бронебойност II (клиновидни, тежки остриета; намаля абсорбацията на ¼)}
%\item{трудно за вадене (не може да се извади от пробита проня по време на битка; не може да се извади от небронирана плът, ако зарът за щета е бил положителен; при вадене лечитеслтво с/у щетите; ако не успее се  нанасят още толкова щети, колкото първоначално)}
%\item{поваляне (обикновено алебарди; наместо щети, атаката може да бъде обявена, че поваля опонента. Ако целта е конник в качествено бойно седло, -5 на атака)}
%\item{обезоръжаване (обикновено меч или щит; при успешна защита има шанс оръжието на врага да е уловено в комплексния гард или шипове на "бос"-а на щит)}
%\end{enumerate}

\subsection{Специални правила}
\begin{center}
\includegraphics[width=0.5\textwidth]{../images/siege}~
\\[1cm]
\end{center}

Езда
нивото на ползване на умения от гърба на кон е ограничено до нивото на умението “езда”
ръкопашни атаки от и по ездач на галопиращ кон са с двойни щети
атаки от и по ездач на кон с късо оръжие търпят -5 атака

Две оръжия
Едното оръжие се ползва за да нанесе щети на врага, нека наречем него първо. Второто оръжие може да се ползва за създаване на откриване (което добавя половината боравене в това оръжие към боравенето с основното оръжие) или пък независимо, за пласиране на стопиращи удари (в който случай се ползва независимо, но с наказание -5 на атака). Ако се атакува само с едното оръжие няма наказания, разбира се.

Щит
Щитът е направен да е второ оръжие. Като такъв не търпи -5 когато се ползва за защита.
пазене от стрели 0/6 (бъклър) през 2/6(кръгъл викингски) през 4/6(дълъг римски) до 6/6 (павис). Това когато активно се криеш от стрелите, разбира се.
Дървените щитове имат здравина 15, а стоманените 30. Това обикновено няма значерние, но силни или бронебойни атаки може да преодолеят успешно блокиране.
\subsubsection{Кавалерия}
