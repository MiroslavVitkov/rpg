% This file declares a template for player races. Here is an example on how to use it:
%\begin{document}
%\race{Race name}{Point price}{Desctiption}{Strength}{Dexterity}{Mind}{Charisma}{Perks}{Race Skills}
%\end{document}

\usepackage{xparse}
\usepackage{ifmtarg}
\usepackage[T2A]{fontenc}
\usepackage[utf8]{inputenc}

\ExplSyntaxOn
\makeatletter
\newcommand{\race}[9]{
    \par\noindent\textbf{#1\ (#2)}
    \@ifmtarg{#3}{}{\\ #3}             % Maintain consistent style between description provided and not provided cases.
%    \@ifmtarg{#3}{\\}{\\ #3 \\}       % Comment above line and uncomment this for more space.
    \par\noindent{Сила:\ #4}
    \par\noindent{Ловкост:\ #5}
    \par\noindent{Ум:\ #6}
    \par\noindent{Харизма:\ #7}
                             \\
                             \\
    Качества:                \\
    \@ifmtarg{#7}{{\topsep}}{\race_template_perks{#8}}
                             
    Расови\ умения:
    \@ifmtarg{#8}{{\topsep}}{\race_template_skills{#9}}
}
\makeatother

\cs_new:Npn \race_template_perks #1{
    \seq_set_split:Nnn \splitted_seq{;}{#1}
    \begin{itemize}
        \seq_map_inline:Nn \splitted_seq{
            \item ##1
        }
    \end{itemize}
}

\cs_new:Npn \race_template_skills #1{
    \seq_set_split:Nnn \splitted_seq{;}{#1}
    \begin{itemize}
        \seq_map_inline:Nn \splitted_seq{
            \item ##1
        }
    \end{itemize}
}
\ExplSyntaxOff
