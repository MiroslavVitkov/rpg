% This file declares a template for player races. Here is an example on how to use it:
%\begin{document}
%\race{Race name}{Point price}{}{}{example 3}{}{}{}
%\end{document}

\usepackage{xparse}
\usepackage{ifmtarg}
\usepackage[T2A]{fontenc}
\usepackage[utf8]{inputenc}

\ExplSyntaxOn
\makeatletter
\newcommand{\race}[6]{
    \par\noindent\textbf{#1}
    \par\noindent{Цена:\ #2} \\
                             \\
    Точки:                   \\
    \par\noindent{Сила:\ #3}
    \par\noindent{Ловкост:\ #4}
    \par\noindent{Ум:\ #5}
    \par\noindent{Харизма:\ #6}
%    \par\noindent\underline{5\ ниво}
%    \@ifmtarg{#7}{\par\vspace{\topsep}}{\list_examples:n{#7}}
%    \par\noindent\underline{6\ ниво}
%    \@ifmtarg{#8}{\par\vspace{\topsep}}{\list_examples:n{#8}}
}
\makeatother

%\cs_new:Npn \list_examples:n #1{
%    \seq_set_split:Nnn \splitted_seq{;}{#1}
%    \begin{itemize}
%        \seq_map_inline:Nn \splitted_seq{
%            \item ##1
%        }
%    \end{itemize}
%}
\ExplSyntaxOff
